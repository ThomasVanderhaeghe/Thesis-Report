\chapter{Conclusion}
In the wake of recent trends in the electricity markets like liberalisation and emission reduction, DER like solar PV and residential batteries are experiencing periods of rapid growth and cost reduction. To further enhance the adoption of these technologies, governments have put up incentive programs to encourage households to adopt renewable energy technologies in a residential setting. These include annual net metering, annual capacity offtake, net billing and battery subsidies. The effects of these incentives on the households' behavior are, however, a major point of ongoing research. This is mainly because the simulation of households' behavior requires the arduous task of incorporating human factors like loss aversion and irrational decision making. This Thesis bridges the gap between the ongoing research in DER adoption modeling, which mainly uses EUT, and the novel concept in behavioral economics that is cumulative prospect theory (CPT).
These concepts from behavioral economics will be combined with an agent-based modeling (ABM) framework to simulate the behavior of the system, which is assumed to be a complex adaptive system (CAS). To that end, this Thesis attempts to perform a more accurate simulation of the behavior of households in the DER adoption process. By examining the DER adoption and distribution tariff data for the different policies, insights about the effectiveness of the different policies can be provided. With these insights, policy makers can identify what policies they want to put in place as a function of their priorities with regards to energy policy. 
\newline \newline \noindent
An optimization model was developed to determine the optimal costs and revenues of different PV/battery configurations. These results were be interpreted by a model that incorporates the behavioral factors of CPT like loss aversion and the reference situation of the households. According to these results, households adopted certain configurations. This adoption causes the aggregate energy demand to decrease, putting the revenue of the DSO at risk. This will force the DSO to increase the network charge, causing the savings for the household to increase, creating a vicious circle of increasing savings and adoption. This model was tested for four cases: (i) annual net metering, (ii) annual capacity offtake, (iii) net billing and (iv) battery subsidies. In this setting, the main Research Question and two sub-questions were answered.
\newline \newline  \noindent
\textbf{Research question 1: What is the effect of annual net metering, annual offtake capacity tariff, net billing, feed-in tariffs, and subsidies on both households' DERs adoption under uncertainty and the utility death spiral?}
\newline \newline \noindent
As became clear from the results discussion, different policies have different benefits. If the policy put in place by the government intends to encourage the rapid adoption of small-scale DER that mainly serve the households' residential demand without injecting vast amounts of electricity into the grid, the net billing policy is a good candidate. This policy will cause the adoption to kick off earlier and happen more quickly. The overall adopted capacity, however, will be rather low. This has both benefits as well as drawbacks. Since the adoption by households in the net billing case happens mainly with the intention of supplying the residential demand without injecting vast amounts of excess electricity into the grid, net billing will cause less stresses onto the grid. Since the frequency and voltage stability issues due to grid injection by intermittent RES are also a major concern in the energy transition, net billing can serve as an effective policy to avoid these grid stability issues indirectly. The overall adopted capacity and produced energy by the DER will, however, be lower in the case of the net billing policy. The overall RES penetration will, therefore, be lower. 
\newline \newline \noindent 
If the policy makers intend to maximize the cumulative adopted capacity of RES, both in PV and battery technology, the net metering policy is more suitable. Since this policy incentivizes households to adopt larger installations, the overall capacity will be visibly larger. This will, however, cause the grid injection to be larger. The grid stability could, therefore, deteriorate under a net metering policy. If policy makers anticipate this and intend to encourage widespread injection into the grid, which is unlikely, net metering is the appropriate policy. 
\newline \newline \noindent
If policy makers want to increase the rate of battery adoption, battery subsidies can be introduced. Although this will not cause the adopted battery capacity to increase exponentially, this incentive can have a visible effect on the batteries, since this technology has not yet had the same adoption and popularity as PV, causing the costs to remain high. The analysis in this Thesis showed that the battery subsidies do not need to be extremely large and that the effect of increasing the subsidies will have a saturating effect on the battery adoption. To limit waste of government resources, therefore, these subsidy programs must be closely monitored. 
\newline \newline \noindent
In addition to the main Research Question of this Thesis, two complementary sub-questions were also part of this research. 
\newline \newline \noindent
\textbf{Research sub-question 1: What is the effect of risk aversion, peer influence and network charges on both the DERs adoption on a household level and the utility death spiral?}
\newline \newline \noindent
For all these policies, the utility death spiral will manifest itself. The annual capacity offtake policy, however, will limit the extent of the utility death spiral. Due to lower savings for the households, however, this policy will cause slower adoption.  Depending on the residual load, the extent of the utility death spiral will vary: the lower the residual load, the less reserve revenue the DSO can rely on and the more the distribution tariff will have to be adapted. Consequently, the lower residual load will cause the DER adoption to happen more quickly. The analysis that has been performed in this Thesis, therefore, was able to show how the DSO should beware of this effect for the volumetric distribution tariff. A possible solution to this problem, as was also analysed in this Thesis, is the introduction of the annual capacity offtake tariff. Due to the different approach of this tariff structure (i.e. as a function of the peak capacity instead of the electricity consumption), the tariff will have to be changed to a lesser extent for DER adoption. Since the tariff will increase less quickly, the savings will be be lower than in the volumetric tariff case. This will cause the subsequent DER adoption to be lower. 
\newline \newline \noindent
\textbf{Research sub-question 2: What is the added value of using a cumulative prospect theory to model investment decision making, compared to expected utility theory?}
\newline \newline \noindent
With regards to the implementation of CPT into this model, the loss aversion sensitivity analysis that was discussed clearly showed how the loss aversion variation causes the adoption to be delayed or occur earlier. This analysis showed the consistency of the results, making a case for further research into this decision-making theory for the simulation of DER adoption. Compared to the EUT results, the added complexity and uncertainty concerning CPT parameters like loss aversion and reference levels also showed to pay off, since the adoption data of the EUT showed much cruder patterns.
\newline \newline \noindent
In future work, the utility attributes can be extended beyond the economical and social one that were considered in this Thesis. Since earlier work (using the EUT) considered additional factors like advertisement effects and household income effects, these can be integrated into the developed model to further refine the model and capture additional behavioral factors. This item has two sub-items that also need further addressing in future work.
\newline \newline \noindent
First of all, since the utility was calculated as a weighted average of social and economical utility using constant coefficients, these coefficients were treated as exogenous variables. In reality, however, these coefficients will change over time, since economical considerations no longer prevail once widespread adoption has been reached, which was assumed in this model. By capturing this shift in weighting coefficients, the model will become even more representative of the adoption process. 
\newline \newline
Secondly, the economic assumptions with regards to the household can be further elaborated upon. Since the model in this Thesis only considers the financial wealth of the households as a proxy for the variation in residential electricity demand, many additional factors can be considered, such as the household size, household age etc. An elaboration of the economic characteristics of the agents in the model could, therefore, also refine the model even further. 
\newline \newline \noindent
In addition, with the knowledge gathered in this Thesis concerning the different policies, these policies can be combined to determine the optimal combination of these different programs in order to maximize the benefits of the different programs, like combining the adoption speed of the net billing policy with the limited utility death spiral of the annual capacity offtake. In doing so, the negative effects of these respective policies must be minimized with regards to all actors in the DER fueled grid transformation process. Furthermore, the sensitivity analysis must be elaborated to more parameters, most notably the time preference of the decision maker. 
\newline \newline \noindent
As a final point of further work, the qualitative study of this model must be extended to a case study taking into account the insights provided in this model. This will allow for the model to be set up under the correct assumptions to simulate the behavior in a real world case. To measure accurate values of the CPT parameters, a field experiment where energy prosumers are the target could also add significant value to this field of research. 





